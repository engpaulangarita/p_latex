\documentclass{article}

\usepackage{courier}
\usepackage[
  height=9in,      % height of the text block
  width=7in,       % width of the text block
  top=78pt,        % distance of the text block from the top of the page
  headheight=48pt, % height for the header block
  headsep=12pt,    % distance from the header block to the text block
  heightrounded,   % ensure an integer number of lines
  showframe,       % show the main blocks
  verbose,         % show the values of the parameters in the log file
]{geometry}
\usepackage{fancyhdr}
\usepackage[utf8]{inputenc}
\usepackage{enumerate}
\usepackage{enumitem}


\pagestyle{fancy}
\fancyhf{} % clear all fields

\fancyhead[L]{\bf{UNIVERSIDAD DEL VALLE - ZARZAL \\ Programa de tecnología en sistemas \\ Introducción a la tecnología en informática}}
\fancyhead[R]{Jueves, diciembre 15, 2016 \\ Semestre II\\  }
\fancyfoot[C]{\thepage}
\fancyfoot[R]{\bf{\textit{Ing. Paul Alexander Angarita Jiménez - 3148446307}}}
\renewcommand{\headrulewidth}{0pt}

\usepackage{lipsum}

\begin{document}
\centerline{\bf  \texttt{Examen Parcial final de introducción a la tecnología informática}} 

\centerline{\bf  \texttt{Saludos estudiantes de ITI versi\'on 2016.}}
El presente parcial está diseñado para evaluar a los estudiantes del curso introducción a la tecnología informática del segundo semestre del 2016 en la universidad del Valle, sede Zarzal.

\section{\textit{Parte I. 60 \% \_\_\_\_\_\_\_\_}}




1.   No es una topología empleada en redes de computadores

\begin{enumerate}[label=(\Alph*)]
\item ad hoc
\item star topology
\item linux topology
\item hibrid
\end{enumerate}

2.  Es un tipo de archivos de gráficos visto en clase

\begin{enumerate}[label=(\Alph*)]
\item .pnj
\item .mex
\item .eps
\item .grafx
\end{enumerate} 
 
3. La siguiente topología utiliza un sistema de broadcasting que envía en particular un dato que es recibido por todos y vuelve a si mismo
\begin{enumerate}[label=(\Alph*)]
\item bus
\item hibrid
\item star
\item token ring
\end{enumerate} 

4. Es una compañia orientada al software de edition
\begin{enumerate}[label=(\Alph*)]
\item Cisco
\item Oracle
\item Adobe
\item Soft edition inc
\end{enumerate} 

5. El siguiente es un entorno para realizar gráficas en latex

\begin{enumerate}[label=(\Alph*)]
\item \textbackslash begin \{grafics\}
\item   \textbackslash begin \{figure\}
\item  \textbackslash begin \{grph\}
\item  \textbackslash begin \{newImage\}



\end{enumerate} 

6. Es un software empleado en la tecnología VOIP


\begin{enumerate}[label=(\Alph*)]
\item packet Tracer
\item linux
\item ekaga
\item Ninguno de los anteriores
\end{enumerate} 

7. Soporta la implementación de conexión entre equipos de la misma capa

\begin{enumerate}[label=(\Alph*)]
\item Cable uno a uno
\item TIA 568 o 568
\item cable en fibra de vidrio
\item Ninguna de los anteriores
\end{enumerate} 


8. Es una de las licencias más conocidas en el mundo del software

\begin{enumerate}[label=(\Alph*)]
\item GNU
\item microsoft
\item InterLicense
\item Ninguna de las anteriores
\end{enumerate} 


9. Permite la simulación de dispositivos de red  

\begin{enumerate}[label=(\Alph*)]
\item Netwage simulator
\item Packet Tracer
\item Ekaga
\item Ninguna de los anteriores
\end{enumerate} 

10. No ha sido el nombre de una referencia o compañía de hardware o procesadores

\begin{enumerate}[label=(\Alph*)]
\item Cyrix Processors
\item Itanium Processors
\item Quantium Processors
\item Athlon Processors
\end{enumerate} 



\section{\textit{Parte II. 40 \% \_\_\_\_\_\_\_\_}}

Realice una guía de una prática de laboratorio en física de un tema específico. Consulte la NUBE del curso la guía y escoja un tema diferente al de su compañero. Debe subirlo a la NUBE del curso, para el lunes, diciembre 19 del 2016. Si desea el presente parcial en LATEX esta disponible para descargar en GitHub.com









%\lipsum[1-2]
\end{document}